

\documentclass{article}
%\setkeys{acs}{usetitle = true}
% Use the postscript times font!
%\usepackage{times}
%\usepackage{graphicx}
%\usepackage{comment}
%\usepackage{amsmath, amssymb, amsthm}
%\usepackage{subfigure}
%\usepackage{algorithm}
%\usepackage [noend]{algorithmic} 

%\usepackage[]{todonotes} % add [disable] before to disable

\usepackage{authblk}


\title{How AI is Helping Synthetic Biology}

% figure out the order later

\author[1]{Fusun Yaman}
\author[2]{Aaron Adler}
\author[3]{Mohammed Eslami}
\author[4]{Jesse Tordoff}
\affil[1]{Raytheon BBN Technologies, 10 Moulton St, Cambridge MA, USA}
\affil[2]{Raytheon BBN Technologies, 9861 Broken Land Parkway, Suite 400, Columbia, MD, USA}
\affil[3]{Netrias}
\affil[4]{Massachusetts Institute of Technology, Cambridge, MA}
\date{}                     %% if you don't need date to appear
\setcounter{Maxaffil}{0}
\renewcommand\Affilfont{\itshape\small}



\begin{document}
  \maketitle
  
  
\begin{abstract}

The abstract.

\end{abstract}

\section{Introduction}  

Introduction

For the AI Magazine article � I think we focus a chunk on what is syn bio and what are the problems that need solving. Maybe specific examples of AI in action in helping address these problems.  I think here we want to highlight some of the hard problems where AI has helped and how we want the AI community to be more involved? Here I think we want to pitch it as a domain with lots of ways for AI people to get involved. 

I like this, I wonder if it's a summary of our discussion session tipped on the challenges that syn bio brings. We had like 6-ish listed if i recall. 


\bibliography{main}

\end{document}

