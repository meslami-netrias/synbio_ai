

\documentclass[journal = asbcd6, manuscript = article]{achemso}
%\setkeys{acs}{usetitle = true}
% Use the postscript times font!
%\usepackage{times}
%\usepackage{graphicx}
%\usepackage{comment}
%\usepackage{amsmath, amssymb, amsthm}
%\usepackage{subfigure}
%\usepackage{algorithm}
%\usepackage [noend]{algorithmic} 

%\usepackage[]{todonotes} % add [disable] before to disable


% figure out the order later
\author{Fusun Yaman}
\email{fusun.yaman@raytheon.com}
\affiliation[Raytheon BBN Technologies]
{10 Moulton St, Cambridge MA, USA}

\author{Aaron Adler}
\email{aaronadler@alum.mit.edu}
\affiliation[Raytheon BBN Technologies]
{9861 Broken Land Parkway, Suite 400, Columbia, MD, USA}

\author{Mohammed Eslami}
\email{meslami@netrias.com}
\affiliation[Netrias]
{Address Goes Here}

\author{Jesse Tordoff}
\email{?@mit.edu}
\affiliation[Massachusetts Institute of Technology]
{I Don't Know, Cambridge MA, USA}

\title
{AI -- Techniques and Applications to Advance SynBio}

\begin{document}

\maketitle

\begin{abstract}

The abstract.

\end{abstract}

\section{Introduction}  

Introduction

For the ACS SynBio article � I think we want a good background primer on AI � maybe with strengths and weaknesses of the different techniques? I guess what I was hoping for in the intro talk. Then I think we�d want to talk about how AI is helping in Synthetic biology. Maybe we can cite drapa programs where AI is helping? I think we want to tackle it from the angle of how to make progress in synbio using AI. So how AI is a tool that can be really useful to Synbio and outline different AI techniques that would apply in different circumstances. Maybe we can throw in something about blindly applying AI / ML being bad (garbage in, garbage out). 

Same comment as above but tipped on the AI side. 
I like this, I wonder if it's a summary of our discussion session tipped on the challenges that syn bio brings. We had like 6-ish listed if i recall. 

\begin{acknowledgement}

Acks.

\end{acknowledgement}

\bibliography{main}

\end{document}

